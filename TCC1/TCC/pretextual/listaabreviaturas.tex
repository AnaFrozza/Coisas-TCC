% quando a sigla for de língua portuguesa utilize \sigla{SIGLA}{Significado em português}
% quando a sigla for de língua estrangeira utilize \siglaIt{SIGLA}{Significado em Inglês}

\sigla{UTFPR}{Universidade Tecnológica Federal do Paraná}
% \siglaIt{ACM}{\textiAssociation for Computing Machinery}
\siglaIt{IP}{Internet Protocol}
\siglaIt{MIDI}{Musical Instruments Digital Interface}
\siglaIt{PCP}{Pitch Class Profiles }
\siglaIt{FFT}{Fast Fourier Transform}

% No texto quando for utilizar a sigla utilize os seguintes comandos:
%\acrlong{label} - acronimo/sigla longo
%\acrshort{label} - acronimo/sigla curta
%\Gls{TCP} - sigla com o significado primeiro em Maiusculo
%\GLS{TCP} - sigla com o significado tudo em MAIUSCULO
%\gls{TCP} - sigla com o significado tudo em minusculo